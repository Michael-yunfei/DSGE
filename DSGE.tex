\documentclass[12pt]{article}
% add some essential packages, some might not be used

\usepackage[T1]{fontenc}
\usepackage[utf8]{inputenc}
\usepackage[usenames,dvipsnames]{color}
\usepackage{natbib}
\usepackage{authblk}
\usepackage{ragged2e}
\usepackage{amsmath}
\usepackage[a4paper,margin=1in,bottom=1.0in]{geometry}
\usepackage{url}
\usepackage{array}
\usepackage{bbding}
\usepackage{amssymb}
\usepackage{graphicx}  % mini page function
\usepackage{adjustbox}
\usepackage{subcaption}
\usepackage{booktabs}
\usepackage{float}
\usepackage{appendix} % appendix package
\usepackage{hyperref}
\usepackage{url}
\usepackage[english]{babel}
\usepackage{adjustbox}
\usepackage{enumitem}
\usepackage{textgreek}
\usepackage{bibentry}
\nobibliography*
\usepackage{lipsum}


\usepackage{listings}
\usepackage{wasysym}
\usepackage{amsthm}
\usepackage{framed}
\usepackage{bm}
\usepackage{booktabs}  % package for table line
% \usepackage{amsrefs?}  % ams citation style package


\usepackage{rotating} % for the horizontal page table

\usepackage{tikz}
\usetikzlibrary{calc}
\usetikzlibrary{matrix}
\usetikzlibrary{positioning}
\usepackage{color}
\usepackage{setspace}
\usepackage{xcolor}

\usepackage{tcolorbox} % package for making colorful box

 \setlength{\parskip}{0.15cm} % change the paragraph spacing
\renewcommand\labelitemi{$\vcenter{\hbox{\tiny$\bullet$}}$} % set the bullet size as tiny

% \newcommand*\rot{\rotatebox{90}} % for rotate text

\usepackage{sectsty} %package for section size

\sectionfont{\fontsize{14}{12}\selectfont} % Change the section font size

\subsectionfont{\fontsize{13}{12}\selectfont}
\subsubsectionfont{\fontsize{12}{12}\selectfont}

\newcommand\numberthis{\addtocounter{equation}{1}\tag{\theequation}} % new command



\theoremstyle{definition}
\newtheorem{definition}[subsubsection]{Definition}
\newtheorem{axiom}[subsection]{Axiom}
\newtheorem{example}[subsubsection]{Example}
\newtheorem{theorem}[subsubsection]{Theorem}
\newtheorem{proposition}[subsubsection]{Proposition}
\newtheorem{lemma}[subsubsection]{Lemma}


\usepackage{courier}

% tikzsetting

\usetikzlibrary{shapes,decorations,arrows,calc,arrows.meta,fit,positioning}

\tikzset{
    -Latex,auto,node distance =1 cm and 1 cm,semithick,
    state/.style ={ellipse, draw, minimum width = 0.7 cm},
    point/.style = {circle, draw, inner sep=0.04cm,fill,node contents={}},
    bidirected/.style={Latex-Latex,dashed},
    el/.style = {inner sep=2pt, align=left, sloped}
}

\lstset{language=Python}

\definecolor{mygreen}{rgb}{0,0.6,0}
\definecolor{mygray}{rgb}{0.5,0.5,0.5}
\definecolor{mymauve}{rgb}{0.58,0,0.82}

\lstset{
  backgroundcolor=\color{white},   % choose the background color; you must add \usepackage{color} or \usepackage{xcolor}; should come as last argument
  basicstyle=\footnotesize,        % the size of the fonts that are used for the code
  breakatwhitespace=false,         % sets if automatic breaks should only happen at whitespace
  breaklines=true,                 % sets automatic line breaking
  captionpos=b,                    % sets the caption-position to bottom
  commentstyle=\color{mygreen},    % comment style
  deletekeywords={...},            % if you want to delete keywords from the given language
  escapeinside={\%*}{*)},          % if you want to add LaTeX within your code
  extendedchars=true,              % lets you use non-ASCII characters; for 8-bits encodings only, does not work with UTF-8
  frame=single,	                   % adds a frame around the code
  keepspaces=true,                 % keeps spaces in text, useful for keeping indentation of code (possibly needs columns=flexible)
  keywordstyle=\color{RoyalBlue},       % keyword style
  language=Python,                 % the language of the code
  morekeywords={*,...},            % if you want to add more keywords to the set
  numbers=left,                    % where to put the line-numbers; possible values are (none, left, right)
  numbersep=5pt,                   % how far the line-numbers are from the code
  numberstyle=\tiny\color{mygray}, % the style that is used for the line-numbers
  rulecolor=\color{black},         % if not set, the frame-color may be changed on line-breaks within not-black text (e.g. comments (green here))
  showspaces=false,                % show spaces everywhere adding particular underscores; it overrides 'showstringspaces'
  showstringspaces=false,          % underline spaces within strings only
  showtabs=false,                  % show tabs within strings adding particular underscores
  stepnumber=2,                    % the step between two line-numbers. If it's 1, each line will be numbered
  stringstyle=\color{mymauve},     % string literal style
  tabsize=2,	                   % sets default tabsize to 2 spaces
  title=\lstname                   % show the filename of files included with \lstinputlisting; also try caption instead of title
}

\numberwithin{equation}{section}
\numberwithin{figure}{section}
\numberwithin{table}{section}


% Define colors
\definecolor{cmd}{HTML}{F7F7F9}
\DeclareMathOperator{\di}{d\!}

\newcommand{\pr}{$\mathbb{P}$}
\newcommand{\pre}{\mathbb{P}}

\begin{document}

\title{DSGE Notes}
\author{Michael}
\date{May, 2019}

\maketitle

\thispagestyle{empty}
\newpage

\section{Review of RBC Model}
\setcounter{page}{1}

I will briefly present the RBC model rather than justifying the model by analysing the main assumptions. For more detailed introduction, please refer to \cite{acemoglu2012introduction}. The model assumes: only one sector of goods exists in the market, the labor is normalised to unit value, agents are homogeneous. The RBC model is also called `neoclassical growth model` or `Ramsey model`.

The intuition(or question) behind this model is: how much of it's income should a nation save or how much I should consume and save everyday to maximise my lifespan utility by enjoying life and meanwhile improving my productivity?

When we talk about business cycle, we focus on the short term and ignore long-run growth. That's why we need apply the filter to get rid of trend. We follow the book by \cite{miao2014economic} and consider the following social planner's problem:
\begin{align}
  \max_{C_t, N_t, I_t} E \sum_{t=0}^\infty \beta^t [\ln C_t + \chi \ln (1 - N_t)]
\end{align}
subject to
\begin{align}
  & C_t + I_t = z_t K_t^\alpha N_t^{1-\alpha} \\
  & K_{t+1} = (1 - \delta) K_t + I_t \\
  & \ln z_t = \rho \ln z_{t-1} + \sigma \varepsilon_t, \varepsilon_t \sim IID \ N(0, 1)
\end{align}
where $\chi$ measures the utility weight on leisure, and the technology shock $\{ z_t \}$ is modelled as an AR(1) process. The optimal allocation $\{ C_t, N_t, K_{t+1}, I_t, Y_t \}$ satisfies the following system of nonlinear difference equations:
\begin{align}
  & \frac{1}{C_t} = E_t \frac{\beta}{C_{t+1}} [\alpha z_{t+1} K_{t+1]}^{\alpha - 1} N_{t+1}^{1-\alpha} + 1 - \delta] \\
  & \frac{\chi C_t}{1 - N_t} = (1 - \alpha) z_t K_t^\alpha N_t^{-\alpha}
\end{align}
with conditions in (1,2) and (1.3). In a decentralized equilibirum in which households own captial and make real investment, the rental rate and the wage rate are given by
\begin{align*}
  R_{kt} = \alpha z_t K_t^{\alpha - 1} N_t^{1-\alpha} \ \ and \ \ w_t = (1-\alpha) z_t K_t^\alpha N_t^{-\alpha}
\end{align*}

In steady state, we have the following system of equations:
\begin{gather*}
  1 = \beta [\alpha K^{\alpha - 1}N^{1-\alpha} + 1 - \delta] \\
\rightarrow \   \frac{K}{N} = \bigg ( \frac{\alpha}{1/\beta - (1 - \delta)} \bigg)^{1/(1-\alpha)} \\
  R_{kt} = \alpha \big (\frac{K}{N}\big)^{\alpha - 1} \\
  w_t = (1-\alpha) \frac{Y}{N} = (1 - \alpha) \big (\frac{K}{N}\big)^\alpha  \\
\end{gather*}
To assign the parameter values into our model, we need examine the long term time series data to derive the stylized facts.

























\newpage
\bibliography{dsge.bib}
\bibliographystyle{apalike}
\end{document}
